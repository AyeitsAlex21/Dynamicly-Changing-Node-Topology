\section{Introduction}

\subsection{Motivation}

\indent The exponential growth of client-server distributed applications has led to a need for them to be scalable systems able to service many users. Many applications currently rely on static configurations or topologies, which could limit their ability to change scale and adapt to changes in traffic. This inflexibility potential could cause increased latency and even service outages. This is because static topology configurations can perform well overall depending on the situation; however; for the internet, there is not a one size fits all solution, and a simple switch of topology to compensate for weaknesses of the previous could lead to a tremendous performance and robustness increase. For example, popular client-server distributed applications like Zoom struggled during the early COVID-19 pandemic because their current infrastructure limited their scalability, fault-tolerance, and efficiency during the surge of the increased nodes and geographical distribution of nodes. Ultimately, this problem that plagued the video calling platforms was fixed using edge computing and CDNs; however, they could have scaled by changing topology.

\indent This leads to the question: How does switching the nodes' topology affect the performance without negatively impacting the user's experience in a distributed client-server application? It is imperative to address this problem to ensure the continued growth of client-server distributed applications to maintain user satisfaction. 

\indent If this study can conclude that switching a topology could provide performance boosts for an application, it opens up the way for developing solutions to dynamically manage the topology's structure to leverage those performance boosts. Therefore, increasing the performance and scalability of future applications. In addition, the implications of this study could be the increased use of client-server technology in distributed applications, as businesses may be interested in its increased performance and reliability. It could also contribute to the understanding and previously unknown insights into client-server networks. Lastly, It could be of value to researchers working on related topics leading to even more innovation in the future.

\subsection{Methods of Changing Topologies}

\indent At the current moment, topology is reconfigurable through the use of an Optical Transport Network (OTN). Software-Defined Networking (SDN) and optical switches can change these fiber connections. SDNs are part of the network stack that decides the routing for packets for each router which, therefore, can change the topology of a network by simply changing the routes for packets. OTN is a protocol that allows optical networks a reliable and efficient way of transporting data across a network. Furthermore, optical switches can change the topology by directly switching one optical signal to another. Thus, this study will explore the switching of topologies under these methods.

\subsection{Outline}



\begin{figure}[tp]
\centering
\includegraphics{figures/mouse}
\caption{\blindtext}
\end{figure}

\section{Related Work}
%\blindtext

And we need some citation here\cite{floyd1993random, stoica2001chord}

%\Blindtext

\section{System Design}

\subsection{The First Layer}

\subsection{The Second Layer}
%\Blindtext

\section{Evaluation}
%\Blindtext

\section{Conclusion}
%\blindtext

